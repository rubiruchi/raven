In the past decade several advanced Dynamic Probabilistic Risk Analysis 
(PRA) methods have been developed. These methods couple stochastic methods
(e.g., RAVEN, ADAPT, ADS, MCDET) with safety analysis codes (e.g.,
RELAP5-3D, MELCOR, MAAP) to determine risk associated to complex systems
such as nuclear plants. Compared to classical PRA methods, which are
based on static Boolean logic structures (e.g., Event-Trees -ET-, Fault-Trees -FT-), 
they can determine risk associated to complex systems with higher resolution 
since the implicitly model timing and sequencing of events
on the accident progression.
In the first part of this article we have presented a set of risk importance 
measures that can be employed on a dataset generated by any Dynamic PRA method
to rank components based on their importance from a safety point of view.
Such measures have been developed as a natural extension of ones employed 
in industry PRA codes and they have been tested on several analytical test cases.
In this second part of the article we show a full comparison between classical and 
Dynamic PRA methods. The system considered for the comparison is a Pressurized
Water Reactor (PWR) system for a large break Loss Of Coolant Accident (LOCA)
initiating event.
We show how the Dynamic and the ET/FT models have been built from both a stochastic and 
accident progression point of view. 
Metrics of comparison are based not only on the core damage probability but also
based on the importance values associated to each basic event.