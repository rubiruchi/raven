\section{Introduction}
\label{sec:introduction}

In the past decades, several numerical simulation codes have been employed 
to simulate accident dynamics, e.g., RELAP5-3D~\cite{relap5}, MELCOR~\cite{Melcor} or MAAP~\cite{maap}. 
In order to evaluate the impact of uncertainties into accident dynamics, 
several stochastic methodologies have been coupled with these codes. 
These stochastic methods range from classical Monte-Carlo and Latin Hypercube 
sampling to stochastic polynomial methods. 
Similar approaches have been introduced into the risk and safety community 
where stochastic methods, e.g., RAVEN~\cite{RAVEN_PSAM_2014}, ADAPT~\cite{ADAPT}, MCDET~\cite{MCDET} or ADS~\cite{ADS}, 
have been coupled with safety analysis codes in order to evaluate the safety 
impact of timing and sequencing of events on the accident progression. 
These approaches are usually called Dynamic PRA methods~\cite{DEVOOGHT_DynamicPRA}.   

Compared to classical PRA methods, which are
based on static Boolean logic structures (e.g., Event-Trees -ET-, Fault-Trees -FT-), 
they can determine risk associated to complex systems with higher resolution 
since the implicitly model timing and sequencing of events
on the accident progression. In ET/FT based methods accident progression is fixed
and it is set prior the analysis by the analyst.
Such approximation is even more limiting for accident scenarios in which tight timing 
dependencies of events are coupled with plant dynamics (e.g., recovery actions and human 
related actions).

The scope of this paper is to present a comparison between Dynamic and classical PRA
methods. The system considered is a Pressurized Water Rector (PWR) for a large break 
Loss Of Coolant Accident (LOCA), LB-LOCA, initiating event.

The comparison Dynamic and classical PRA
methods is based not only on the core damage probability but also
based on the importance values associated to each basic event.

In~\cite{RIM_part1} we have presented a set of risk importance 
measures that can be employed on a dataset generated by any Dynamic PRA method
to rank components based on their importance from a safety point of view.
Such measures have been developed as a natural extension of the ones employed
in industry PRA codes such SAPHIRE~\cite{saphire} or CAFTA~\cite{cafta}.

In this paper we employs such metrics extensively in order to quantify such comparison. 
The objectives are twofold: the first one is to validate the proposed metrics over 
an industry-grade test case and show capabilities of Dynamic PRA methods. 
The second one is to provide guidance on how it is possible to have classical and dynamic
PRA methods compatible with each other. The rationale behind it is that a Dynamic PRA
analysis can be performed for a limited aspect of the overall system and its result 
incorporated into a classical PRA results. Similarly, a classical PRA analysis can be 
``made dynamic'' by adding dynamic elements (e.g., quantitative information of timing 
of events) into it and coupled with a system simulation code.