\section{Comparison of Results }
\label{sec:comparison}

The comparison between the two PRA analyses has been performed on three different steps by considering:
\begin{enumerate}
	\item CD probability
	\item Probability associated to each accident progression sequence
	\item Importance measure (FV and RAW) associate to each macro basic event
\end{enumerate}
Note how the three steps have an increasing level of detail from coarse (i.e., first step) to very fine 
(i.e., importance measures).
This approach allows us to progressively measure the differences between classical and Dynamic PRA methods.

Regarding the first step, the comparison of $P_{CD}$ shows very similar results as shown in 
Table~\ref{tab:case1_P_CD}. Hence, as first conclusion it can be stated that the accident progression has been
modeled in a fairly identical way.

The next analysis step has focused on the set of accident progression sequences generated by SAPHIRE in and 
ET form and the ones generated by RAVEN/RELAP5-3D in the form of simulated transients.
This has been performed by associating each transient simulated by RAVEN/RELAP5-3D to a specific branch of the 
LOCA ET as follows:
\begin{itemize}
  \item identify the ET branching conditions in the transient temporal evolution (e.g., successful activation of 
        the accumulator system)
  \item determine the successful/unsuccessful outcome of each branching condition
  \item identify the ET branch that matches the set of branching condition outcomes; if not match is found then
        the ET requires a review (e.g., additional branches/branching conditions)
\end{itemize}

Figure~\ref{} shows a summary of this analysis where, for each ET branch, the corresponding branch probability 
has been calculated using SAPHIRE and RAVEN/RELAP5-3D.
The major difference in this analysis is located in the probability associated to branch 4 

\begin{table}
  \caption{Comparison of results: analysis of $P_{CD}$}
  \label{tab:case1_P_CD}
  \centering
  \begin{tabular}{c | c | c } 
    \hline 
             & SAPHIRE & RAVEN \\ 
    \hline 
    $P_{CD}$ & 1.0     & 1.0   \\
    \hline 
  \end{tabular}
\end{table}


\begin{table}
	\centering
	\label{tab:case1_RIMs}
	\caption{Comparison of results.}
	\begin{tabular}{*5c}
		\hline
		BE &  \multicolumn{2}{c}{SAPHIRE}     & \multicolumn{2}{c}{RAVEN}\\
		{}               &  FV        &  RAW       &  FV        &  RAW        \\
		\hline
	    IE-LLOCA         &  1.        &  4.00 E+5  &  FV        & RAW  \\ 
	    LPR-XHE-XM-ERROR &  8.63 E-1  &  123       &  8.50      & 121  \\
	    LPI-SYS-CF-DEM   &  8.82 E-2  &  123       &  8.62 E-2  & 121  \\ 
	    LPR-SYS-CF-DEM   &  6.67 E-3  &  123       &  1.25 E-3  & 121  \\ 
	    LPI-SYS-CF-RUN   &  1.29 E-3  &  123       &  1.25 E-3  & 121  \\
	    LPR-SYS-CF-RUN   &  5.25 E-3  &  123       &  1.25 E-3  & 121  \\
	    NSW-SYS-CF-DEM   &  4.92 E-4  &  123       &  4.8 E-4   & 121  \\
	    ACC-CKV-CC-CL1   &  1.97 E-3  &  68.9      &  0         & 1  \\
	    ACC-CKV-CC-CL2   &  1.97 E-3  &  68.9      &  0         & 1  \\
	    ACC-CKV-CC-CL3   &  1.97 E-3  &  68.9      &  0         & 1  \\
	    LPI-SYS-DEM-TRNA &  9.15 E-3  &  3.57      &  8.72 E-3  & 3.48  \\
	    LPI-SYS-DEM-TRNB &  9.15 E-3  &  3.57      &  8.72 E-3  & 3.48  \\
	    LPR-SYS-DEM-TRNA &  1.10 E-2  &  3.56      &  1.04 E-2  & 3.48  \\
	    LPR-SYS-DEM-TRNB &  1.10 E-2  &  3.56      &  1.04 E-2  & 3.48  \\
	    LPI-SYS-RUN-TRNA &  1.37 E-3  &  3.51      &  1.33 E-3  & 3.48  \\
	    LPI-SYS-RUN-TRNB &  1.37 E-3  &  3.51      &  1.33 E-3  & 3.48  \\
	    LPI-SYS-TM-TRNA  &  8.12 E-3  &  2.00      &  2.0 E-2   & 3.48  \\
	    LPI-SYS-TM-TRNB  &  8.12 E-3  &  2.00      &  2.0 E-2   & 3.48  \\
	    LPR-SYS-TM-TRNA  &  5.07 E-3  &  2.00      &  1.24 E-2  & 3.48  \\
	    LPR-SYS-TM-TRNB  &  5.07 E-3  &  2.00      &  1.24 E-2  & 3.48  \\
		\hline
	\end{tabular}
\end{table}


\subsection{LLOCA success criteria [Carlo]}

[text]

