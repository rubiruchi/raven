\section{Conclusions}
\label{sec:conclusions}

This report summarizes the R&D activities performed during the FY2016 regarding 
data mining to time-dependent data. We have structured this development to 
leverage the existing RAVEN capabilities (i.e., data mining of static data) and 
include new algorithms and methods ad-hoc to analyze time-dependent data.

We have indicated several data analysis directions that can be employed to analyze 
time-dependent data. It is important to highlight that these directions are not valid 
only for RISMC type of applications. Regarding RISMC type of applications we have shown 
several test cases that employed both analytical models but also nuclear industry codes 
such as RELAP5-3D and MAAP.
The objective of these test cases was to provide possible analysis direction when 
dealing with nuclear transients. We have shown how it was possible to create a mapping 
between:
\begin{itemize}
  \item simulation temporal profile
  \item simulation outcome
  \item timing and sequencing of events. 
\end{itemize}

This is the kind of information that is relevant from a PRA point of view (and thus 
also a RISMC one).
It is relevant to highlight that, even though it was not explicitly shown in this report, 
such analysis methods and algorithms were successfully employed to detect and resolve 
issues and errors located in the model and in the data generation step. 
Some of these issues that could have not been detected with classical data analysis methods 
were for example:
\begin{itemize}
  \item Erroneous implementation of the system dynamic (e.g., wrong simulation time step)
  \item Erroneous implementation of the system control logic (e.g., wrong activation threshold 
        of components)
  \item Wrong implementation of the data management of the statistical framework during the 
        data generation phase (e.g., Monte-Carlo sampling or Dynamic Event Tree)
\end{itemize}

Thus, such analysis greatly improved the quality and the understanding of the data generated. 
Classical methods to perform data analysis would have not been able to reach such a level of 
data exploration.
In this report we have also shown how it was possible to measure reliability importance of 
components in a simulation based PRA environment (such as RISMC). The analyses presented 
in this report resemble the ones performed using risk-importance measures in a classical 
PRA environment (such as SPAR models). In our case we focused the attention more the link 
between timing/sequencing of events and sampled variables with simulation temporal profile 
without explicitly considering any probabilistic type of information.
In conclusion, the development of the RAVEN code during FY16 has provide precious capabilities 
to the code itself in terms of time-dependent data mining that no other software product 
(commercial or open-source) can offer. These capacities must be looked not only from a RISMC 
(or in general simulation-based PRA) perspective but also to a more broad engineering 
perspective that includes: scenario forecasting (predictive capabilities), code calibration 
on time-dependent data etc. 


