Simulation-Based Probabilistic Risk Assessment (PRA) methods employs
system simulation codes coupled with stochastic analysis tools
in order to determine probabilities of certain outcomes such as 
system failure.
In contrast to classical PRA methods (i.e., Event-Tree and Fault-Tree)
in which timing and sequencing of events is set by the analyst,
accident progression is dictated by the system control logic 
and its interaction with system temporal evolution.
Due to the nature of the problem, Simulation-Based PRA methods require 
large amount of computationally expensive simulation 
runs. 
In addition, these methods usually generate a 
large number of simulation runs (database storage may be on the order of 
gigabytes or higher). 
In this paper we investigate and apply several methods 
and algorithms to analyze these large amounts of time-dependent data. 
The scope of this article is to present a broad overview of methods and 
algorithms that can be used to improve data quality and to analyze and extract information from large 
data sets containing time dependent data. In this context, “extracting information” 
means constructing input-output correlations, finding commonalities, and identifying 
outliers. 