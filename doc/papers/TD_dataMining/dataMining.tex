\section{Data Mining}
\label{sec:dataMining}

Data mining is a fairly generic concept that entails the generation of information/knowledge 
from data sets.  The process of generation of information/knowledge can be performed in various 
ways depending on the type of application but it possible to classify all data analysis approaches 
into four categories:
\begin{itemize}
  \item ROM: ROM algorithms aim to reduce to the complexity of the data by 
        finding a mathematical objects that emulate the behavior of the data by learning its input/output 
        relations and reconstructing such relations through a regression/interpolation based approach
  \item Dimensionality reduction: this category includes all methods than aim to reduce the dimensionality 
        of the data set and project the original data into a reduced space
  \item Clustering: algorithms in this category partition the data based on a set of defined similarity measures
  \item Data searching: identify the closest data point in a database given a reference point. Once that 
        point has been found it is possible to infer properties of the reference point
\end{itemize} 

The range of applications that are of interest from an engineering point of view (and, hence, not 
limited to a typical RISMC application) are the following:
\begin{itemize}
  \item Analysis of data input-output relationship: this is the most relevant application within the RISMC 
        project, i.e., the identification of the connection between input parameters (which are stochastically 
        sampled) and the full temporal profile of the simulation. Using the system of equation (??), we want 
        to create a correlation between s and θ(t). While it is trivial to make a connection
        between input sampled parameters and static output parameters, such as max clad temperature or 
        average fluid speed, it gets tougher when we want to consider the full temporal profile of the 
        simulation
  \item Outlier identification: due to numerical limitations of system simulator codes (e.g., range of 
        validity for correlations), the outcome of a subset of simulation runs may contain wrong results. 
        When these limitations are reached, the temporal profile may contain anomalous behaviors. The
        objective of data mining would be to identify these anomalous behaviors, i.e. outliers. This would 
        dramatically increase the quality of the data being generated/analyzed
  \item Diagnosis/prognosis: while this application is not referenced in detail in this work, it is worth 
        highlighting that all data mining methods presented in this work can be used in an operating environment 
        to assist reactor operators to assess plant conditions and possible plant evolutions
        during an accident scenario. The basic idea would be use a large set of simulations generated for different 
        accident conditions as a database with the objective to constantly match the actual plant status. 
        Once a match has been established (i.e., a set of simulated scenarios match closely enough the actual and 
        past plant status) the operators can now:
        \begin{itemize}
          \item identify the possible causes that is generating the abnormal behavior in the plant (diagnosis)
          \item predict possible future evolutions of the accident scenarios (prognosis)
        \end{itemize} 
\end{itemize} 

In order to maintain the same standard throughout the paper, we introduce here the set of mathematical symbols 
and terminology. The types of data that are considered in this report are exclusively numerical data, i.e., 
data which consists of numbers of any format (either integer or double)
\footnote{This paper does not consider symbolic data sets, i.e., data which consists of symbols (e.g., text data)}. 
Given this,
we can represent a generic data set $\Xi$ as a collection of $N$ objects:
\begin{equation}
  \Xi = \{ H_1,\ldots,H_N \}  
  \label{eq:collection}
\end{equation}
Each object $H_n$ $(n=1,\ldots,N)$ lies in a multi-dimensional space $\Pi$: $H_n \in \Pi$ and $\Xi \subset \Pi$. 
From now on we will consider $\Pi$ 
as a metric space~\cite{}: $\Pi$ is a space where it is possible to define an arbitrary distance function, i.e., a 
metric.
Since we are dealing with numerical data in a multi-dimensional space we can say that $\Xi \subset \mathbb{R}$ 
where $D$ is the dimensionality of the space:
\begin{equation}
  H_n=[H_n^1,\ldots,H_n^d,\ldots,H_n^D ]
  \label{eq:collection_upd}
\end{equation}
We indicate with $x^d  (d=1,\ldots,D)$ each coordinate $d$ of $\Xi \subset \mathbb{R}$. 
Thus, each element $H_n^d$ of $H_n$ is the 
$d$ coordinate, i.e. dimension $x^d (d=1,\ldots,D)$, of the data point $H_n$.

\begin{figure}
  \centering
  \begin{subfigure}{.5\textwidth}
    \centering
    \includegraphics[scale=0.3]{hier_1.png}
    \label{fig:sub1}
  \end{subfigure}%
  \begin{subfigure}{.5\textwidth}
    \centering
    \includegraphics[scale=0.3]{hier_3.png}
    \label{fig:ABsystem}
  \end{subfigure}
  \caption{Components A and B in a series configuration (left) and its associated Fault-Tree (right).}
  \label{fig:ABsystem}
\end{figure}

\section{Clustering Algorithms}
\label{sec:clusteringAlgorithms}

From a mathematical viewpoint, clustering~\cite{SurveyClustering} aims is to find a partition
\begin{math} \mathbf{C}=\{C_{1},\ldots,C_{l},\ldots,{C_{L}}\}\end{math}
of $\Xi$ where each $C_{l}$ $(l=1,\ldots,L)$ is called a cluster. The partition
    $ \mathbf{C} $ of $ \mathbf{X} $
is such that:
    \begin{equation}\label{eq: ClassRequier}
        \begin{cases} \mathbf{C}_{l}\neq\varnothing, l=1,\ldots,L \\
                        \\
                     \bigcup_{l=1}^{L}\mathbf{C}_{l}= \Xi \\
        \end{cases}
    \end{equation}

Even though the number of clustering algorithms available in the literature is large, usually the most 
used ones when applied to time series are the following: Hierarchical~\cite{}, 
K-Means~\cite{} and Mean-shift~\cite{}.
Hierarchical algorithms build a hierarchical tree from the individual points (leaves) by progressively 
merging them into clusters until all points are inside a single cluster (root). 
Clustering algorithms such as K-Means and Mean-Shift, on the other hand, seek a single partition of 
the data sets instead of a nested sequence of partitions obtained by hierarchical methodologies. 

\subsection{Hierarchical}
\label{sec:hierarchical}

The hierarchical clustering is, along with the K-Means one (see Section~\ref{sec:kmeans}), the most basic 
clustering algorithms available in literature. Hierarchical algorithms organize data into a 
structure according to a proximity matrix in which each element $(j,k)$ is some measure of the 
similarity (or distance) between the items to which row $j$ and column $k$ correspond. 
Usually, the final result of these algorithms is a binary tree, also called dendrogram, 
in which the root of the tree represents the whole data set and each leaf is a data point. 
To show how hierarchical clustering works we will employ the data set pictured in 
Figure~\ref{}, i.e., a data set containing about one hundred points in a 2-dimensional space. 

Given a set of $N$ items to be clustered, and a $N \times N$ distance (or dissimilarity) matrix $\Delta$, 
the basic steps 
behind hierarchical clustering are the following:
\begin{enumerate}
  \item Assign each data point to a cluster, i.e., from $N$ data points, $N$ clusters are initialized, 
        each cluster contains just one data point. The distances among each cluster is the same as the
        distances between the data points they contain
  \item Determine the closest pair of clusters and merge them into a single cluster
  \item Determine the distances between the new cluster (obtained in Step 2) and the remaining clusters
  \item Repeat steps 2 and 3 until all items are clustered into a single cluster of size $N$
\end{enumerate}

\begin{figure}
    \centering
    \includegraphics[scale=0.2]{hier_2.png}
    \caption{}
    \label{fig:2Danalogy}
\end{figure}
  
\subsection{K-Means}
\label{sec:kMeans}  
  
The K-Means algorithm~\cite{} is probably the most simple clustering algorithm. The basic idea is to 
determine $K$ centers ($K$ is provided as an input variable) $\chi_k (k=1,\ldots,K)$: one center for each cluster.
At the beginning these centers are placed randomly in the feature space. The  next  step is to take 
each point of the data set and associate it to the nearest center $\chi_k$. At this point the centers $\chi_k$ 
are re-calculated as average of the data points associated to $\chi_k$. This data points and centers average 
step is repeated. As a result of this loop, the K centers change their location step by step until no 
more changes are done or, in other words, centers do not move any more. 

\subsection{Mean-Shift}
\label{sec:meanShift}

Mode-seeking approaches~\cite{} look at the density distribution of data points lying in a metric space. 
Clusters are viewed as regions of the space with high point density separated by regions of low
point density. Clusters can be identified by searching for regions of high density, called modes. 
For the comparison a Mode-seeking algorithm referred to as the Mean-Shift.

The Mean-Shift algorithm~\cite{} is a non-parametric iterative procedure that can be used to assign each 
point to one cluster center through a set of local averaging operations. The local averaging operations 
provide empirical cluster centers within the locality and define the vector which denotes the direction 
of increase for the underlying unknown density function.

The main idea is to consider each data point ??? of the dataset as an empirical 
distribution density function, or kernel, $K(\vec{x})$ distributed in a multidimensional space where 
regions with high data density (i.e., modes) correspond to local maxima of the multivariate kernel 
density estimate $f_{I}(\vec{x})$ \cite{CacoullosEstimation} (see Fig.~\ref{fig:densityfunc}).


