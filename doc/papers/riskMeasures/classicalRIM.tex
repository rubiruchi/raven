\section{Classical RIMs}
\label{sec:classicalRIMs}

In classical PRA methods, for any basic event, the most used RIMs measures are: 
Risk Achievement Worth (RAW), Risk Reduction Worth (RRW), Birnbaum (B) and 
Fussell-Vesely (FV)~\cite{flemingRiskImportance}. 
All these RIMs are calculated by determining three values based on core damage 
frequency (CDF):
\begin{itemize}
  \item $R_0$: nominal CDF
  \item $R_i^-$: CDF for basic event i assuming perfectly reliable
  \item $R_i^+$: CDF for basic event i assuming it has failed
\end{itemize}

Once these three values are determined, then the RIMs are calculated~\cite{} as 
follows for each basic event $i$:
\begin{align} 
  RAW_i &= \frac{R_i^+}{R_0}  \\
  RRW_i &= \frac{R_0}{R_i^-}  \\
  B_i &= R_i^+-R_i^-   \\
  FV_i &= \frac{R_0-R_i^-}{R_0} 
\end{align}

Note the four RIMs listed above is not exhaustive: in literature it is possible to 
find additional RIMs such as the 
Differential Importance Measure (DIM)~\cite{BorgonovoApostolakis}. 
Since, the scope of this paper is tight to risk-informed application of 10CFR50.69, 
we will focus this paper only on the four RIMs listed above.

