\section{Introduction}
\label{sec:introduction}

Risk Importance Measures (RIMs)~\cite{Nureg3385} are indexes that are used to rank systems, 
structures and components (SSCs) based on their contribution to the overall risk. 
The most used measures~\cite{flemingRiskImportance} are: 
Risk Reduction Worth (RRW), Risk Achievement Worth (RAW), Birnbaum (B) and Fussell-Vesely (FV).

Typically, this ranking is performed in a classical PRA framework, where risk is determined by 
considering probability associated to the minimal cut-sets generated by static logic 
structures~\cite{Nureg1150} (e.g., Event-Trees, Fault-Trees). In a classical PRA analysis, 
each SSC is represented by a set of basic events; as an example emergency diesel generators can be 
represented by two basic events: failure to start and failure to run. 

The risk measures associated to each basic event are calculated from the generated cut-sets 
by determining: 
\begin{itemize}
  \item The nominal risk 
  \item The increased risk assuming basic event failed
  \item The reduced risk assuming basic event perfectly reliable 
\end{itemize}

In this context, the Nuclear Regulatory Commission (NRC) has issued the 10CFR50.69 
document~\cite{} designed to guide plant owners to perform a risk-informed categorization and 
treatment of SSCs in order to reduce operating and maintenance costs while preserving 
acceptable risk levels. The described categorization is based on a set of risk importance 
measures obtained from the plant classic PRA models.

In contrast to classical PRA methods, Dynamic PRA methods~\cite{DEVOOGHT_DynamicPRA} couple 
stochastic methods 
(e.g., RAVEN~\cite{RAVEN_PSAM_2014}, ADAPT~\cite{ADAPT}, ADS~\cite{ADS}, MCDET~\cite{MCDET}) 
with safety analysis 
codes (RELAP5-3D~\cite{relap5}, MELCOR~\cite{Melcor}, MAAP~\cite{maap}) to determine risk 
associate to 
complex systems such as nuclear plants. Accident progression is thus determined by the 
simulation code and not set a-priori by the user. The advantage of this approach, 
compared to classical PRA methods, is that a higher fidelity of the results can be achieved since:
\begin{itemize}
  \item No assumption of timing/sequencing of events is set by the user but dictated by the 
        accident evolution
  \item No success criteria are defined but instead, the simulation stops if either a fail 
        or a success state are reached
  \item There is no need to compute convolution integrals in order to specify probability of 
        basic events that temporally depends from other basic events.
\end{itemize}

The scope of this paper is to present a method to determine classical RIMs from Dynamic PRA 
data. Few test cases will be presented in order to show how the calculation is performed. 
In addition, new margin-centric RIMs that better capture the continuous aspect of a Dynamic 
PRA approach will be presented.

